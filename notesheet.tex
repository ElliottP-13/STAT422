\documentclass[10pt]{article}
\usepackage{./EllioStyle}
\usepackage{listings}
\usepackage{multicol}

\definecolor{codegreen}{rgb}{0,0.6,0}
\definecolor{codegray}{rgb}{0.5,0.5,0.5}
\definecolor{codepurple}{rgb}{0.58,0,0.82}
\definecolor{backcolour}{rgb}{0.95,0.95,0.92}

\usepackage[small]{titlesec}
\newtheorem{definition}{Definition}[section]

\lstdefinestyle{mystyle}{
%    backgroundcolor=\color{backcolour},   
    commentstyle=\color{codegreen},
    keywordstyle=\color{magenta},
    numberstyle=\tiny\color{codegray},
    stringstyle=\color{codepurple},
    basicstyle=\ttfamily\footnotesize,
    breakatwhitespace=false,         
    breaklines=true,                 
    captionpos=b,                    
    keepspaces=true,                 
    numbers=left,                    
    numbersep=5pt,                  
    showspaces=false,                
    showstringspaces=false,
    showtabs=false,                  
    tabsize=2
}

\titlespacing{\section}{0pt}{0pt}{0pt}
\setlength{\parskip}{8pt}

\title{Notesheet}
\author{Elliott Pryor}
\date{February 11}

\rhead{Notesheet}

\begin{document}
\multicols{2}


\section{Distributions}
Uniform: $f(x) = \frac{1}{b - a}; a \leq x \leq b$;\\
$\bar{X} = \frac{a + b}{2}$;
$Var(x) = \frac{(b-a)^2}{12}$\\
$mgf = \frac{e^{tb} - e^{ta}}{t(b - a)}$

Normal: $f(x) = \frac{1}{\sigma \sqrt{2 \pi}} \exp{\frac{-1}{2\sigma^2}(x - \mu)^2}$;\\
$\bar{X} = \mu$;
$Var(x) = \sigma^2$;\\
$mgf = \exp{\mu t + \frac{t^2 \sigma^2}{2}}$;

Exponential: $f(x) = \frac{1}{\beta} e^{\frac{-x}{\beta}}$;\\
$\bar{X} = \beta$;
$Var(x) = \beta^2$;\\
$mgf = (1 = \beta t)^{-1}$;

Gamma: $f(x) = \frac{1}{\Gamma(\alpha)\beta^\alpha} x^{\alpha - 1} e^{\frac{-x}{\beta}}$;\\
$\bar{X} = \alpha \beta$;
$Var(x) = \alpha \beta^2$;\\
$mgf = (1 = \beta t)^{-\alpha}$;
$\Gamma(\alpha) = \int_0^\infty y^{\alpha - 1}e^{-y}dy$

Beta: $f(x) = \frac{\Gamma(\alpha + \beta)}{\Gamma(\alpha) \Gamma(\beta)} x^{\alpha-1} (1 - x)^{\beta - 1}$;\\
$\bar{X} = \frac{\alpha}{\alpha + \beta}$;\\
$Var(x) = \frac{\alpha \beta}{(\alpha + \beta)^2(\alpha + \beta + 1)}$;


\section{Misc}
\emph{Moment Generating Function}:\\
$m_{X_1, \dots X_n} = \mathbb{E}[e^{tX_1 + \dots + tX_n}] \stackrel{iid}{=} \prod_{i=1}^n \mathbb{E}[e^{tX_i}]$

Jacobian Method: $Y = g(X)$ with $g(X)$ monotone:\\
$f_Y(y) = f_X(g^{-1}) \left| \frac{d}{dy} g^{-1}(y) \right|$.\\
If $g(X)$ not monotone, then $f_Y$ is sum of piecewise monotone $g_i(X)$.

\section{Order Statistics}
Maximum:
$F_{X_{(n)}}(x) = F_X(x) ^n$, \\ 
$f_{X_{(n)}} = n (F_X(x))^{n-1}f_X(x)$ 

Minimum:
$F_{X_{(1)}}(x) = 1 - (1 - F_X(x))^n$, \\
$f_{X_{(1)}}(x) = n (1 - F_X(x)) ^ {n-1} f_X(x)$

General:\\
$f_{X_{(j)}} = \frac{n!}{(j-1)!(n-j)!}F_X(x)^{j-1}f_X(x)(1-F_X(x))^{n-j}$


\section{Normal, Chi-Square Distributions}
Mean: $\bar{X} \sim N(\mu, \frac{\sigma^2}{n})$.

Given $X \sim N(\mu, \sigma^2)$, $Z_i = \frac{X_i - \mu}{\sigma}$. Then:\\
$\sum_{i=1}^n Z_i^2 \sim \chi^2_n$

$\chi^2_\nu = Gamma(\frac{\nu}{2}, \beta=2)$, $\mathbb{E}[X] = \nu, Var(x) = 2\nu$.

Theorem 7.3: $\bar{x}, S^2$ are independent, and:
$$\frac{(n - 1)S^2}{\sigma^2} \sim \chi^2_{n-1} \quad \quad S^2 = \frac{1}{n-1} \sum_{i=1}^n (x_i - \bar{x})^2$$

Student T Distribution: if $Z \sim N(0, 1)$ and $W \sim \chi^2_\nu$, and $Z, W$ independent. Then RV:
$T = \frac{Z}{\sqrt{W / \nu}}$ follows the student T distribution. Where:\\
$$f_T(x) = \frac{\Gamma(\frac{\nu + 1}{2})}{\Gamma(\frac{\nu}{2}) \sqrt{\pi \nu}}(1 + \frac{t^2}{\nu})^{-(\nu + 1)/2}$$


Snedecor's F Distribution: if $U \sim \chi^2 _u$ and $V \sim \chi^2 _v$ and $U,V$ independent.
Then $X = \frac{U / u}{V / v} \sim F_{u,v}$. Where:
$$f_X(x; u, v) = \frac{\Gamma(\frac{u + v}{2})}{\Gamma(u/2)(v/2)} (\frac{u}{v}^{u/2}) x^{u/2 - 1} (1 + \frac{u}{v} * x)^{-(u + v)/2} \quad x > 0$$

\section{Central Limit Theorem}
Let $X_1, \dots, X_n$ be iid RV with expected value $\mu$ and variance $\sigma^2$ and assume that mgf exists.
The cdf of $\bar{X}$ converges to a normal distribution as $n \to \infty$. 
$Z_n = (\bar{X} - \mu)/(\sigma / \sqrt{n})$, then $\lim_{n \to \infty} P(Z_n \leq z) = P(Z \leq z)$ where $Z \sim N(0,1)$.
I.e. the sampling distribution for the sample mean $\overset{a}{\sim} N(\mu, \sigma^2/n)$

\section{Method of Moments}
Goal is to match the pointwise moments to the distribution moments.
Need one equation for each unknown.
Set $\mathbb{E}[X] = \frac{1}{n} \sum x_i$, $\mathbb{E}[X^2] = \frac{1}{n} \sum x_i ^2$, $\dots$ for however many equations needed.

\section{Maximum Likelihood Extimators}
Compute likelihood function $\mathcal{L}(\theta | x) = \text{joint distribution} = f(x | \theta)$.
Find $\hat{\theta}_{MLE}$ that maximizes $\mathcal{L}(\theta | x)$. 
Solve by setting $\frac{\partial }{ \partial \theta_i} L(\theta | x) = 0$.
Can also compute log-likelihood which is easier as $l(\theta | x) = \ln (L(\theta | x))$.
Don't forget! Verify that critical point is the maximum by verifying second derivative at critical point $< 0$.

\section{Mean Squared Error and Bias}
$Bias_\theta(\hat{\theta}) = \mathbb{E}[\hat{\theta}] - \theta$.
An estimator is said to be unbiased if $\mathbb{E}[\hat{\theta}] = \theta$.
The mean squared error is: 
$MSE_\theta(\hat(\theta)) = \mathbb{E}_\theta[(\hat{\theta} - \theta)^2] = Var_\theta(\hat{\theta}) + Bias_\theta(\hat{\theta})^2$.
MSE combines accuracy and precision into one measure, if MSE increases the estimator is `less good'.

Best unbiased estimator - Uniform Minimum Variance Unbiased Estimator (UMVUE) is $Bias_\theta (\hat{\theta}^*) = 0$ and 
$Var(\hat{\theta}^*) \leq Var(\hat{\theta}) \quad \hat{\theta} \in \{ \hat{\theta} | Bias_\theta(\hat{\theta}) = 0 \}$.


\section{Sufficiency, Completeness, UMVUE}
\begin{definition}[UMVUE]
    An estimator $\hat{\theta}$ is UMVUE if: 
    $Bias_\theta(\hat{\theta}) = 0$ and 
    $Var(\hat{\theta}) \leq Var(\theta') \quad \forall \theta' \in \{\theta' | Bias_\theta(\theta') = 0\}$
\end{definition}
\begin{definition} [Sufficient]
    Let $X_1, \dots, X_n$ be a random sample from a probability distribution with parameter $\theta$,
    The statistic, $T(X)$ is said to be sufficient for $\theta$ if the conditional distribution of $(X_1, \dots, X_n) | T(X)$
    does not depend on $\theta$.
\end{definition}
Factorization Theorem: let $f(x; \theta)$ be joint pdf of $X_1, \dots, X_n$. 
A statistic is sufficient for $\theta$ is sufficient iff $\exists g(T(x); \theta), h(x)$ such that:
$f(x; \theta) = g(T(x) ; \theta) * h(x)$. 
\begin{definition}[Complete Statistic]
    A statistic is complete if $\mathbb{E}[g(T)] = 0 \quad \forall \theta$ implies that $P(g(T) = 0) = 1 \quad \forall \theta$
\end{definition}
Exponential Family Theorem: Let $X_1, \dots, X_n$ be an iid RS from a pdf of the form
$$f(x;\theta) = h(x) c(\theta) \exp \left( \sum_{j=1}^k t_j(x) w_j(\theta) \right)$$
Then the set of statistics $T(X) = \sum_{i=1}^n t_1(X_i), \dots, \sum_{i=1}^n t_k(X_i)$ is complete and sufficient for $\theta$.





\end{document}